\documentclass[12pt,a4paper]{article}
\usepackage[utf8]{inputenc}
\usepackage[russian]{babel}
\usepackage{amsmath,amssymb,amsthm}
\usepackage[margin=1in]{geometry}
\usepackage{mathtools}
\usepackage{enumitem}
\usepackage{titlesec}
\usepackage{xcolor}
\usepackage{hyperref}

% Настройка заголовков
\titleformat{\section}
  {\normalfont\Large\bfseries\color{blue}}{\thesection}{1em}{}

\titleformat{\subsection}
  {\normalfont\large\bfseries\color{blue}}{\thesubsection}{1em}{}

% Определение цвета для выделений
\definecolor{myblue}{RGB}{25, 70, 145}
\definecolor{mygreen}{RGB}{40, 150, 50}

% Установка гиперссылок
\hypersetup{
    colorlinks=true,
    linkcolor=myblue,
    urlcolor=myblue
}

% Оформление задачи
\newcommand{\zadacha}[2]{\section*{Задача #1} \textcolor{myblue}{\textbf{#2}}}

\author{\textbf{Асписов Дмитрий Алексеевич, БПИ226}}
\date{\today}
\title{\textbf{Домашнее задание 2}}

\begin{document}

\maketitle

\zadacha{1}{Найдите решение системы сравнений:}
\[
\begin{cases}
x \equiv 3 \ (\text{mod} \ 5) \\
x \equiv 4 \ (\text{mod} \ 6) \\
x \equiv 5 \ (\text{mod} \ 7)
\end{cases}
\]

\subsection*{Решение:}

1. Находим значения произведений модулей \(M_i\):
\[
M = 5 \times 6 \times 7 = 210
\]
\[
M_1 = \frac{M}{5} = 42, \quad M_2 = \frac{M}{6} = 35, \quad M_3 = \frac{M}{7} = 30
\]

2. Решаем для \(N_i\) из уравнений:
\[
\begin{aligned}
&42 \times N_1 \equiv 1\ (\text{mod} \ 5) \quad \Rightarrow 2 \times N_1 \equiv 1\ (\text{mod} \ 5) \quad \Rightarrow N_1 = 3, \\
&35 \times N_2 \equiv 1\ (\text{mod} \ 6) \quad \Rightarrow 5 \times N_2 \equiv 1\ (\text{mod} \ 6) \quad \Rightarrow N_2 = 5, \\
&30 \times N_3 \equiv 1\ (\text{mod} \ 7) \quad \Rightarrow 2 \times N_3 \equiv 1\ (\text{mod} \ 7) \quad \Rightarrow N_3 = 4.
\end{aligned}
\]

3. Находим \(x\):
\[
x = (3 \cdot 42 \cdot 3) + (4 \cdot 35 \cdot 5) + (5 \cdot 30 \cdot 4) = 378 + 700 + 600 = 1678
\]
Запишем решение в виде сравнения по модулю:
\[
x \equiv 1678 \ (\text{mod} \ 210)
\]
\[
\boxed{x \equiv 208\ (\text{mod}\ 210)}
\]

\zadacha{2}{Сколько решений в \(\mathbb{Z}_5\) имеет система \(x \equiv 3y \ (\text{mod} \ 5)\)?}

\subsection*{Решение:}

Рассмотрим все возможные значения \(y\) в \(\mathbb{Z}_5\) и найдём соответствующие \(x\):
\[
\begin{aligned}
&1)\ y = 0 \implies x = 3 \times 0 \equiv 0 \ (\text{mod} \ 5), \\
&2)\ y = 1 \implies x = 3 \times 1 \equiv 3 \ (\text{mod} \ 5), \\
&3)\ y = 2 \implies x = 3 \times 2 \equiv 1 \ (\text{mod} \ 5), \\
&4)\ y = 3 \implies x = 3 \times 3 \equiv 4 \ (\text{mod} \ 5), \\
&5)\ y = 4 \implies x = 3 \times 4 \equiv 2 \ (\text{mod} \ 5).
\end{aligned}
\]
Таким образом, для каждого значения \(y\) существует уникальное значение \(x\). Поскольку \(y\) может принимать 5 различных значений в \(\mathbb{Z}_5\), получаем:
\[
\boxed{5 \text{ решений в } \mathbb{Z}_5}
\]

\zadacha{3}{Может ли при составном \(n\) выполняться сравнение \((n-1)! \equiv -1 \ (\text{mod} \ n)\)?}

\subsection*{Решение:}



\end{document}
