\documentclass[12pt,a4paper]{article}
\usepackage[utf8]{inputenc}
\usepackage[russian]{babel}
\usepackage{amsmath,amssymb,amsthm}
\usepackage[margin=1in]{geometry}
\usepackage{mathtools}
\usepackage{enumitem}
\usepackage{titlesec}
\usepackage{xcolor}
\usepackage{hyperref}

\titleformat{\section}
  {\normalfont\Large\bfseries\color{blue}}{\thesection}{1em}{}
\titleformat{\subsection}
  {\normalfont\large\bfseries\color{blue}}{\thesubsection}{1em}{}

\definecolor{myblue}{RGB}{25, 70, 145}
\definecolor{mygreen}{RGB}{40, 150, 50}

\hypersetup{
    colorlinks=true,
    linkcolor=myblue,
    urlcolor=myblue
}

\newcommand{\zadacha}[2]{\section*{Задача #1} \textcolor{myblue}{\textbf{#2}}}

\author{\textbf{Асписов Дмитрий Алексеевич, БПИ226}}
\date{\today}
\title{\textbf{Домашнее задание 2}}

\begin{document}

\maketitle

\zadacha{4}{Построить вывод секвенции \( \Rightarrow (p \to q) \to ((\neg p \to q) \to q) \)}

\subsection*{Решение:}

\begin{enumerate}
    \item Применяем правило импликации справа ($\to R$):
    \[
    \Rightarrow (p \to q) \to ((\neg p \to q) \to q)
    \]
    превращается в:
    \[
    p \to q \Rightarrow (\neg p \to q) \to q
    \]
    
    \item Применяем правило импликации справа ($\to R$) для второй части:
    \[
    p \to q \Rightarrow \neg p \to q \Rightarrow q
    \]
    
    \item Применяем правило импликации слева ($\to L$) к $p \to q$. Получаем два случая:
    \[
    \{p \Rightarrow q, \neg p \to q\} \quad \text{и} \quad \{\neg p \Rightarrow q, \neg p \to q\}
    \]
    
    \item В первой ветви, секвенция $p \Rightarrow q, \neg p \to q$ выводима по правилу аксиомы ($p \Rightarrow p$), так как $p \to q$ и $p$ дают $q$.

    \item Во второй ветви, секвенция $\neg p \Rightarrow q, \neg p \to q$ также выводима, так как $\neg p$ и $\neg p \to q$ дают $q$ через импликацию.
\end{enumerate}

Таким образом, вывод секвенции завершён.



\zadacha{5}{Доказать, что секвенция \( \Rightarrow (p \to q) \to \neg p \) невыводима}

\subsection*{Решение:}

\begin{enumerate}
    \item Рассмотрим структуру секвенции. Необходимо показать, что из $p \to q$ следует $\neg p$.
    \item Применим правило импликации справа ($\to R$):
    \[
    p \to q \Rightarrow \neg p
    \]
    \item Секвенция $p \to q \Rightarrow \neg p$ утверждает, что если $p \to q$, то $p$ должно быть ложным. Однако, это невозможно вывести, так как:
    \begin{itemize}
        \item $p \to q$ означает лишь, что если $p$ истинно, то $q$ истинно, но это не даёт информации о том, что $p$ ложно.
        \item Для вывода $\neg p$ нам нужно доказать ложность $p$, но $p \to q$ не подразумевает этого.
    \end{itemize}
    \item Следовательно, секвенция невыводима.
\end{enumerate}

Таким образом, данная секвенция не может быть доказана.


\end{document}