\documentclass[12pt,a4paper]{article}
\usepackage[utf8]{inputenc}
\usepackage[russian]{babel}
\usepackage{amsmath,amssymb,amsthm}
\usepackage[margin=1in]{geometry}
\usepackage{mathtools}
\usepackage{enumitem}
\usepackage{titlesec}
\usepackage{xcolor}
\usepackage{hyperref}

\titleformat{\section}
  {\normalfont\Large\bfseries\color{blue}}{\thesection}{1em}{}

\title{\textbf{Домашнее задание 1}}
\author{\textbf{Асписов Дмитрий Алексеевич БПИ226}}
\date{\today}

\begin{document}

\maketitle

\section*{Задача 19}
\textcolor{blue}{\textbf{Приведите к ДНФ формулу:}}
\[
(p \vee q) \rightarrow (p \vee \neg r)
\]

\textbf{Решение:}

\begin{enumerate}[label=\arabic*., leftmargin=*]
\item Преобразуем импликацию:
   \[
   (p \vee q) \rightarrow (p \vee \neg r) 
   \equiv \neg (p \vee q) \vee (p \vee \neg r)
   \]

\item Применим правило Де Моргана:
   \[
   (\neg p \wedge \neg q) \vee (p \vee \neg r)
   \]

\item Применим дистрибутивный закон:
   \[
   (\neg p \vee (p \vee \neg r)) \wedge (\neg q \vee (p \vee \neg r))
   \]

\item Упростим:
   \begin{itemize}
   \item $\neg p \vee (p \vee \neg r) \equiv \top \vee \neg r \equiv \top$
   \item $\neg q \vee (p \vee \neg r)$ остаётся неизменным
   \end{itemize}

\item Итоговая ДНФ:
   \[
   \boxed{\neg q \vee p \vee \neg r}
   \]
\end{enumerate}

\section*{Задача 20}
\textcolor{blue}{\textbf{Докажите, что следующая формула является тавтологией для любого $n$:}}

\[
\bigwedge_{i=1}^{n+1} \bigvee_{j=1}^n p_{ij} \rightarrow \bigvee_{j=1}^n \bigvee_{i_1,i_2=1, i_1<i_2}^{n+1} (p_{i_1j} \wedge p_{i_2j}).
\]

\textbf{Решение:}

Предположим, что матрица \(p\) размером \((n+1) \times n\) содержит элементы из нулей и единиц.

\begin{proof} (от противного)

\begin{enumerate}[label=\arabic*., leftmargin=*]
\item Предположим, что левая часть истинна, а правая — ложна.

\item Левая часть: $\bigwedge_{i=1}^{n+1} \bigvee_{j=1}^n p_{ij}$
   \begin{itemize}
   \item В каждой строке матрицы $p$ есть хотя бы одна единица.
   \end{itemize}

\item Правая часть: $\bigvee_{j=1}^n \bigvee_{i_1, i_2=1, i_1<i_2}^{n+1} (p_{i_1j} \wedge p_{i_2j})$
   \begin{itemize}
   \item Ни в одном столбце нет двух единиц.
   \end{itemize}

\item Расставляем единицы:
   \begin{itemize}
   \item По одной в каждой строке.
   \item Не более одной в каждом столбце.
   \end{itemize}

\item Противоречие:
   \begin{itemize}
   \item Строк $n+1$, столбцов $n$.
   \item После заполнения $n$ столбцов, одна строка останется без единицы.
   \item По принципу Дирихле найдётся столбец с двумя единицами.
   \end{itemize}

\item Вывод: исходное предположение неверно.
\end{enumerate}

Следовательно, формула является тавтологией.
\end{proof}

\section*{Бонусная задача}
\textcolor{blue}{\textbf{Найдите число $n$-местных шефферовых функций, т.е. найдите формулу, которая по числу $n$ даёт число $n$-местных шефферовых функций.}}

\textbf{Решение:}

Шефферова функция — это булева функция, которая может быть выражена с использованием только одной операции: штрих Шеффера (\(\mid\)), которая является отрицанием дизъюнкции.

Операция Шеффера является функционально полной, то есть любые булевы функции могут быть выражены с её помощью.

Чтобы найти количество всех \(n\)-местных шефферовых функций, рассмотрим следующее:

\begin{enumerate}[label=\arabic*., leftmargin=*]
    \item \textbf{Общее количество булевых функций:} 
    \begin{itemize}
        \item Для \(n\)-местной булевой функции существует \(2^n\) различных наборов значений переменных. Для каждого набора переменных результат может быть либо 0, либо 1.
        \item Поэтому общее количество булевых функций на \(n\) переменных равно \(2^{2^n}\). Это количество включает все возможные комбинации значений на выходе для всех наборов входных переменных.
    \end{itemize}
    
    \item \textbf{Шефферова функция:}
    \begin{itemize}
        \item Шефферова функция должна включать хотя бы одну ложь, то есть она не может быть тождественно истинной (функция, которая всегда возвращает 1). Это связано с тем, что штрих Шеффера (\(\mid\)) определён как отрицание дизъюнкции и всегда даёт ложь для некоторых входных значений.
        \item Поэтому из всех булевых функций исключаем одну — тождественно истинную функцию.
    \end{itemize}
    
    \item \textbf{Количество шефферовых функций:} 
    \begin{itemize}
        \item Количество шефферовых функций равно количеству всех булевых функций, за исключением одной тождественно истинной.
        \item Следовательно, общее количество \(n\)-местных шефферовых функций равно \(2^{2^n} - 1\).
    \end{itemize}
\end{enumerate}

Таким образом, количество \(n\)-местных шефферовых функций равно:
\[
\boxed{2^{2^n} - 1}
\]\end{document}
